\documentclass[10pt]{article}
% Headers that are used to control style in document
\usepackage[top=0.75in,bottom=0.5in,left=0.5in,right=0.5in, headsep=0.3in]{geometry}
\usepackage{graphicx}
\usepackage{setspace}
\usepackage{parskip}
\usepackage{fancyhdr}
\usepackage{authblk}
%	Headers for various technical writing aspects. EX: equations, units
\usepackage{amsmath}
\usepackage{siunitx}
\usepackage{listings}
\usepackage{xcolor}
\usepackage{pdfpages}


\definecolor{functions}{HTML}{630090}
\definecolor{comments}{RGB}{80,150,45}
\definecolor{type-defs}{RGB}{40,129,214}
 % Setup the lstlisting for assembly for ARM


  \lstdefinelanguage{ARM}{
    morekeywords=[1]{
			MOV,ADD,SUB, MUL,DIV, AND,ORR,EOR,BIC,LDR,STR,PUSH,POP,BL,BX,B,BEQ,BNE,BGT,BLT,CMP,CMN,TST,TEQ,ADR,SWI
    },
    morekeywords=[2]{
			INCLUDE, AREA, CODE, READONLY, EXPORT, ALIGN, ENTRY, PROC, ENDP, END
    },
    sensitive=true,
    morecomment=[l]{;},
    morestring=[b]"
  }

\lstset{
  % language=ARM,
  language=C,
  basicstyle=\ttfamily\footnotesize,
  keywordstyle=[1]{\color{type-defs}\textbf},
  keywordstyle=[2]\color{red}\textbf,
  morekeywords=[3]{main,setup,ADC_Init,USART_Init,USART_Read,USART_Write,SysTick_Initialize,SysTick_Handler,Delay,delay_ms},
  morekeywords=[4]{int,uint8_t,uint32_t,double,float,USART_TypeDef,char,unsigned,volatile},
  keywordstyle=[3]\color{functions},
  keywordstyle=[4]\color{type-defs},
  commentstyle=\color{comments},
  stringstyle=\color{blue},
  numbers = left,
  numberstyle=\tiny\color{gray},
  frame=single,
  tabsize=4,
  showstringspaces=false
}
% Header setup
	\pagestyle{empty}

% Define the title and authors
	\title{\vspace{-2em}\bf Program 2 Report}
	\author{Kelson Petersen}
	\date{\today}

% Document Start
\begin{document}
\maketitle

\section*{\bf Objectives}
To review programming in C or C++. To give practice in converting mathematical descriptions of ideaas into computer 
implementations. To provide practice and experience in convolution. To provide practice in finding the complete solution of
differential equations.
\section*{\bf Discrete Convolution in C}
To perform convolution in a digital computer it is easier to make a continuous signal discrete. We do this by `sampling' 
the continuous signal at specific points in time. We then are able to convolve that signal with other discrete signals.
The method to convolve two discrete functions is create an array that holds $n\cdot m-1$ elements where $n$ is the length
of signal one, and $m$ is the length of signal two. The convolution is the sum of all diagonal elements in a table of 
products as seen in the scratch work section of the appendix.

To write this code it is fairly simple and only requires the following code.
\lstinputlisting{./../code/convolve.c}
The double nested loop in lines 14\textendash18 iterate through each element of the two arrays passed to the function.
While iterating throught the arrays the index $i+j$ is how we get the diagonal elements to sum together in the correct 
places.
\section*{\bf Conclusion}
Words
\newpage
\section*{Appendix}
Words
\subsection*{Code}
Words
\subsection*{Scratch Work}
\includepdf[pages={1-2}]{./apdx/scratch_prog_2.pdf}
\end{document}
