\documentclass[10pt]{article}
% Headers that are used to control style in document
\usepackage[top=0.75in,bottom=0.5in,left=0.5in,right=0.5in, headsep=0.3in]{geometry}
\usepackage{graphicx}
\usepackage{setspace}
\usepackage{parskip}
\usepackage{fancyhdr}
\usepackage{authblk}
%	Headers for various technical writing aspects. EX: equations, units
\usepackage{amsmath}
\usepackage{siunitx}
\usepackage{listings}
\usepackage{xcolor}
\usepackage{pdfpages}


\definecolor{functions}{HTML}{630090}
\definecolor{comments}{RGB}{80,150,45}
\definecolor{type-defs}{RGB}{40,129,214}
 % Setup the lstlisting for assembly for ARM


  \lstdefinelanguage{ARM}{
    morekeywords=[1]{
			MOV,ADD,SUB, MUL,DIV, AND,ORR,EOR,BIC,LDR,STR,PUSH,POP,BL,BX,B,BEQ,BNE,BGT,BLT,CMP,CMN,TST,TEQ,ADR,SWI
    },
    morekeywords=[2]{
			INCLUDE, AREA, CODE, READONLY, EXPORT, ALIGN, ENTRY, PROC, ENDP, END
    },
    sensitive=true,
    morecomment=[l]{;},
    morestring=[b]"
  }

\lstset{
  % language=ARM,
  language=C,
  basicstyle=\ttfamily\footnotesize,
  keywordstyle=[1]{\color{type-defs}\textbf},
  keywordstyle=[2]\color{red}\textbf,
  morekeywords=[3]{main,setup,ADC_Init,USART_Init,USART_Read,USART_Write,SysTick_Initialize,SysTick_Handler,Delay,delay_ms},
  morekeywords=[4]{int,uint8_t,uint32_t,double,float,USART_TypeDef,char,unsigned,volatile},
  keywordstyle=[3]\color{functions},
  keywordstyle=[4]\color{type-defs},
  commentstyle=\color{comments},
  stringstyle=\color{blue},
  numbers = left,
  numberstyle=\tiny\color{gray},
  frame=single,
  tabsize=4,
  showstringspaces=false
}
% Header setup
	\pagestyle{empty}

% Define the title and authors
	\title{\vspace{-2em}\bf Program 2 Report}
	\author{Kelson Petersen}
	\date{\today}

% Document Start
\begin{document}
\maketitle

\section*{\bf Objectives}
To review programming in C or C++. To give practice in converting mathematical descriptions of ideaas into computer 
implementations. To provide practice and experience in convolution. To provide practice in finding the complete solution of
differential equations.
\section*{\bf Discrete Convolution in C}
To perform convolution in a digital computer it is easier to make a continuous signal discrete. We do this by `sampling' 
the continuous signal at specific points in time. We then are able to convolve that signal with other discrete signals.
The method to convolve two discrete functions is create an array that holds $n\cdot m-1$ elements where $n$ is the length
of signal one, and $m$ is the length of signal two. The convolution is the sum of all diagonal elements in a table of 
products as seen in the scratch work section of the appendix.

To write this code it is fairly simple and only requires the following code.
\lstinputlisting{./../code/convolve.c}
The double nested loop in lines 14\textendash18 iterate through each element of the two arrays passed to the function.
While iterating throught the arrays the index $i+j$ is how we get the diagonal elements to sum together in the correct 
places. 

The results for convolutions $f1*f1, f1*f2, f1*f3, f2*f3,f1*f4$ are given below in the figures section. The program 
included in the appendix prints to the terminal the following.
\begin{lstlisting}
f1:      0     1     2     3     2     1
f2:     -2    -2    -2    -2    -2    -2    -2
f3:      1    -1     1    -1
f4:      0     0     0    -3    -3

f1 * f1:      0     0     1     4    10    16    19    16    10     4     1
f1 * f2:      0    -2    -6   -12   -16   -18   -18   -18   -16   -12    -6    -2
f1 * f3:      0     1     1     2     0     0    -2    -1    -1
f2 * f3:     -2     0    -2     0     0     0     0     2     0     2
f1 * f4:      0     0     0     0    -3    -9   -15   -15    -9    -3%    
\end{lstlisting}
\section*{\bf Total Solution}
Given the equation
\begin{align}
  (D^3+5D^2+12D+15)y(t) = (D+1.5)f(t)
\end{align}
With initial conditions $y(0) = -2,\dot{y}(0) = 3,\ddot{y}(0) = 4$ we find with the scratch work done in the appendix 
that the solution to the above equation is the following:
\begin{align}
  y_{\text{\it\ total}} = \left( -0.07e^{-2.6038t}+1.94e^{-1.19t}\cos(2.08-2.98)+0.0164\cos(7.85t+2.03)\right)u(t)
\end{align}
With the above solution we are able to use the included code to get the graphs in the figures section below.
\section*{\bf Figures}
\begin{figure}[h!]
  \center
  \includegraphics[scale=0.25]{./figs/Final_Plot.png}
  \caption{Program Numeric Solution vs. Analytical Solution}
\end{figure}
\begin{figure}[h!]
  \center
  \includegraphics[scale=0.5]{./figs/f1_f1.png}
  \caption{Convolution of f1 and f1}
\end{figure}
\begin{figure}[h!]
  \center
  \includegraphics[scale=0.5]{./figs/f1_f2.png}
  \caption{Convolution of f1 and f2}
\end{figure}
\begin{figure}[h!]
  \center
  \includegraphics[scale=0.5]{./figs/f1_f3.png}
  \caption{Convolution of f1 and f3}
\end{figure}
\begin{figure}[h!]
  \center
  \includegraphics[scale=0.5]{./figs/f2_f3.png}
  \caption{Convolution of f2 and f3}
\end{figure}
\begin{figure}[h!]
  \center
  \includegraphics[scale=0.5]{./figs/f1_f4.png}
  \caption{Convolution of f1 and f4}
\end{figure}
\newpage
\section*{\bf Conclusion}
The final result ends up settling to a ``steady state'' oscillation with a magnitude of 0.0164. This magnitude is the 
result of the total solution having one part multiplied by a decaying exponential and the other part does not. The part
without the decaying exponential is the resulting steady state output of the solution.
\newpage
\section*{Appendix}
In this section we have the code and the analytical solutions done by hand.
\subsection*{Code}
\lstinputlisting{./../code/total_graph_res.m}
\lstinputlisting{./../code/P1Plots.m}
\lstinputlisting{./../code/main.c}
\lstinputlisting{./../code/matrix.h}
\lstinputlisting{./../code/matrix.c}
\lstinputlisting{./../code/convolve.h}
\lstinputlisting{./../code/convolve.c}
\lstinputlisting{./../code/sine.c}
\lstinputlisting{./../code/h_t.c}
\subsection*{Scratch Work}
This section contains the hand analysis. The following pages contain the hand work.
\includepdf[pages={1-3}]{./apdx/scratch_prog_2.pdf}
\end{document}
