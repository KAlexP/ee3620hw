\documentclass[12pt]{article}

\usepackage[margin=0.5in]{geometry}
\usepackage{xcolor}
\usepackage{listings}
\usepackage{amsmath}
\usepackage{pgfplots}
\usepackage{graphicx}
\usepackage{setspace}
\usepackage{parskip}
\usepackage{hyperref}
\usepackage{pdfpages}

\hypersetup{
 linktoc=all,
 colorlinks=true,
  linkcolor=black
}

\lstset{
  language=c,
  basicstyle=\ttfamily\footnotesize,
  keywordstyle=[1]\color{blue}\textbf,
  keywordstyle=[2]\color{red}\textbf,
  commentstyle=\color{gray},
  stringstyle=\color{red},
  numbers = left,
  numberstyle=\tiny\color{gray},
  frame=single,
  tabsize=4,
  showstringspaces=false
}

\author{Kelson A. Petersen, A02346768}
\date{\today}
\title{Program 1 Report}


\begin{document}
\maketitle

\tableofcontents

\newpage
\section{Problem 1}
\begin{figure}[!h]
  \center
  \includegraphics[scale=0.5]{../figures/fig_1_overlay.png}
  \caption{The Exact solution compared to the numeric approximation of Problem 1.}
\end{figure}

Using figure 1 above we see the C program approximate solution and the exact solution
follow the same curve and we have a good approximation of the first order ODE given 
by $(D + 2.5)y_0(t) = 0$ with the initial condition $y_0(0)=3$. Scratch work is included 
at the end of the document.

The following code is the loop responsible for solving the ODE that was given above. The 
entirety of the code is included later in this document.
\lstinputlisting{./../prob1.c}

\newpage
\section{Problem 2}

\begin{figure}[!h]
  \center
  \includegraphics[scale=0.5]{../figures/fig_2_overlay.png}
  \caption{The Exact solution compared to the numeric approximation in Problem 2.}
\end{figure}
Using figure 2 above we compare the exact solution that is the blue dashed line and the 
approximate solution in the C code below follow a similar path; however, the approximation 
is less accurate than the exact solution in this case. The oscillilatory nature of this 
function causes the `average' to be less accurate. Below is the figure that has the roots 
of the exact solution.

\begin{figure}[!h]
  \center
  \includegraphics[scale=0.5]{../figures/roots.png}
  \caption{The roots of the differential equation for Problem 2.}
\end{figure}

Below is the equation that was found as the solution for the differential equation given by 
$D^3 + 0.6D^3 + 25.1125D + 2.5063y_0(t) = 0$. I used matlab to solve.
\begin{align}
\frac{21156110196\,\sqrt{10009}\,{\mathrm{e}}^{-\frac{3\,t}{10}}\,\sin\left(\frac{\sqrt{10009}\,t}{20}\right)}{5049641841125}-\frac{5221024\,{\mathrm{e}}^{-\frac{3\,t}{10}}\,\cos\left(\frac{\sqrt{10009}\,t}{20}\right)}{504510125}-\frac{25063\,t}{251125}+\frac{1523972423}{1009020250}
\end{align}

This is the specific code that was used to calculate the approximation. 
\lstinputlisting{./../prob2.c}

\newpage
\section{Problem 3}

\begin{figure}[!h]
  \center 
  \includegraphics[scale=0.5]{./../figures/fig_3_overlay.png}
  \caption{The Exact solution compared to the numeric approximation in Problem 3.}
\end{figure}
For the figure above we solved the circuit to find the differential equation that work is shown at the end 
of the document. The differential equation we end up with is $f(t) = D^2LCR_2(\frac{1}{R+1} + \frac{1}{R_2}) + D(\frac{L}{R_1} + CR_2) + y(t)$.
We see the approximate solution worked well with this equation as it does not oscillate. If the circuit had a dependend source in it the 
analytical and numeric solutions would both become more complicated as the non-linearity of the circuit would create 
different frequency responses varying in a non-linear fashion. Additionally The numeric approximation I would expect to see diverge from the 
exact solution because the exact solution would not be linear.

Below is the code that did the calculation.
\lstinputlisting{./../prob3.c}


\newpage
\section{Code}
\lstinputlisting{./../main.c}

\section{Scratch Work}
\includepdf[pages=-]{../Program-Scratch.pdf}

\end{document}
