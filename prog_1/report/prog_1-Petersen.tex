\documentclass[12pt]{article}

\usepackage[margin=0.5in]{geometry}
\usepackage{xcolor}
\usepackage{listings}
\usepackage{amsmath}
\usepackage{pgfplots}

\lstset{
  language=c,
  basicstyle=\ttfamily\footnotesize,
  keywordstyle=[1]\color{blue}\textbf,
  keywordstyle=[2]\color{red}\textbf,
  commentstyle=\color{gray},
  stringstyle=\color{red},
  numbers = left,
  numberstyle=\tiny\color{gray},
  frame=single,
  tabsize=4,
  showstringspaces=false
}

\author{Kelson A. Petersen}
\date{\today}
\title{Program 1 Report}


\begin{document}
\maketitle

\tableofcontents

\newpage
\section{Problem 1}
\begin{figure}[!h]
  \center
  \includegraphics[scale=0.5]{../fig_1_overlay.png}
  \caption{The Exact solution compared to the numeric approximation of Problem 1.}
\end{figure}


\newpage
\section{Problem 2}
\begin{figure}[!h]
  \center
  \includegraphics[scale=0.5]{../fig_2_overlay.png}
  \caption{The Exact solution compared to the numeric approximation in Problem 2.}
\end{figure}


\newpage
\section{Problem 3}


\newpage
\section{Code}

In this section there will be brief descriptions of the code followed by the code.


The first file included here is a file named `main.h'. This file is the header file 
that contains the preview of the function for the compiler allowing the definitions 
of the functions to be in different files.
\lstinputlisting{./../main.h}


The following code is the entry point of the program that simply calls the correct 
function that corresponds to the problem selected by the user. There is basic error 
messages that are printed if the function called fails to create the necessary files.
\lstinputlisting{./../main.c}


The following code is the code solution for problem one. This code was mostly given in
the program assignment description, but the loop creates a file containing x and y 
coordinates to generate a graph.
\lstinputlisting{./../prob1.c}


The following code is the code solution for problem two. This is an extension of the 
code in problem one, but for higher order equations. We are able to use the state-space 
form of a differential equation to make this process work. We put the equation in the 
state-space form by using matrices.
\lstinputlisting{./../prob2.c}


The following code is the code for a differential equation that was derived from a RLC 
circuit in problem three. It is solve the same way as problem 2.
\lstinputlisting{./../prob3.c}


The following code was given in the program 1 description. It is a collection of matrix 
operation functions, and the functions allow the matrix operations.
\lstinputlisting{./../vec_funcs.c}

\lstinputlisting{./../makefile}

\end{document}
